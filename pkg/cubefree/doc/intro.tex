%%%%%%%%%%%%%%%%%%%%%%%%%%%%%%%%%%%%%%%%%%%%%%%%%%%%%%%%%%%%%%%%%%%%%%%%%
%%
%W  intro.tex          Cubefree documentation              Heiko Dietrich
%%
%%

%%%%%%%%%%%%%%%%%%%%%%%%%%%%%%%%%%%%%%%%%%%%%%%%%%%%%%%%%%%%%%%%%%%%%%%%%
\Chapter{Introduction}

\atindex{Cubefree}{@{\Cubefree}}

%%%%%%%%%%%%%%%%%%%%%%%%%%%%%%%%%%%%%%%%%%%%%%%%%%%%%%%%%%%%%%%%%%%%%%%%%
\Section{Overview}

This manual describes the {\Cubefree}  package,
a {\GAP} 4 package for constructing groups of cubefree order; i.e., groups
whose order is not divisible by any third power of a prime.
 
The groups of squarefree order are known for a long time: Hoelder
\cite{Hol93} investigated them at the end of the 19th century. Taunt
\cite{Tau55} has considered solvable groups of cubefree order, since he
examined solvable groups with abelian Sylow subgroups. Cubefree groups in
general are investigated firstly in \cite{Di05}, \cite{DiEi05}, and \cite{DiEi05add}, and this
package contains the implementation of the algorithms described
there.

Some general approaches to construct groups of an arbitrarily given order are
described in \cite{BeEia}, \cite{BeEib}, and \cite{BeEiO}.

The main function of this package is a method to construct 
all groups of a given cubefree order up to isomorphism. The algorithm behind this function is
described completely in \cite{Di05} and \cite{DiEi05}. It is a refinement of
the methods of the {\GrpConst} package which are described in \cite{GrpConst}.

This main function needs a method to construct up to conjugacy the solvable
cubefree subgroups of GL$(2,p)$ coprime to $p$. We split this construction
into the construction of reducible and irreducible subgroups of GL$(2,p)$. To determine the
irreducible subgroups we use the method described in \cite{FlOB05} for which this package
also contains an implementation. Alternatively, the {\IrredSol} package
\cite{Irredsol} could be used for primes $p\le 251$.

The algorithm of \cite{FlOB05} requires a method to rewrite a matrix
representation. We use and implement the method of \cite{GlHo97} for this purpose.

One can modify the construction algorithm for cubefree groups to a very
efficient algorithm to construct groups of squarefree order. This is already
done in the  {\SmallGroups} library. Thus for the construction of groups of squarefree order it is more practical to
use `AllSmallGroups' of the {\SmallGroups} library. 

A more detailed description of the implemented methods can be found in Chapter 2.

Chapter "Installing and Loading the Cubefree Package" explains
how to install and load the {\Cubefree} package.


%%%%%%%%%%%%%%%%%%%%%%%%%%%%%%%%%%%%%%%%%%%%%%%%%%%%%%%%%%%%%%%%%%%%%%%%%
\Section{Theoretical background}

In this section we give a brief survey about the main algorithm which is used
to construct groups of cubefree order: the Frattini extension method. For a by
far more detailed description
we refer to the above references; e.g. see the online version of \cite{Di05}.

Let $G$ be a finite group. The Frattini subgroup $\Phi(G)$ is defined to be
the intersection of all maximal subgroups of $G$. We say a group $H$ is a
Frattini extension by $G$ if the Frattini factor $H/\Phi(H)$ is isomorphic to
$G$. The Frattini factor of $H$ is Frattini-free; i.e. it has a trivial
Frattini subgroup. It is known that every prime divisor of $|H|$ is also a divisor of
$|H/\Phi(H)|$. Thus the Frattini subgroup of a cubefree group has to be
squarefree and, as it is nilpotent, it is a direct product of cyclic groups of
prime order.

Hence in order to construct all groups of a given cubefree order $n$, say, one can,
firstly, construct all Frattini-free groups of suitable orders and, secondly, compute
all corresponding Frattini extensions of order $n$. A first fundamental result
is that  a group of cubefree order is either a solvable Frattini
extension or a direct product of a PSL$(2,r)$, $r>3$ a prime, with a solvable
Frattini extension. In particular, the simple groups of cubefree
order are the groups PSL$(2,r)$ with $r>3$ a prime such that $r\pm
1$ is cubefree. As a nilpotent group is the direct product of its Sylow subgroups, it
is straightforward to compute all nilpotent groups of a given cubefree order.

Another important result is
that for a cubefree solvable Frattini-free group there is exactly one isomorphism
type of suitable Frattini extensions, which restricts the construction of
cubefree groups to the
determination of cubefree solvable
Frattini-free groups.  This uniqueness of Frattini extensions is the
main reason why the Frattini extension method works so efficiently in the
cubefree case. 

In other words, there is a one-to-one correspondence between
the solvable cubefree groups of order $n$ and {\it some} Frattini-free groups of order
dividing $n$. This allows to count the number of isomorphism types of cubefree groups of a given
order without
constructing Frattini extensions.

In the remaining part of this section we consider the construction of the
solvable Frattini-free groups of a given cubefree order up to
isomorphism. Such a group is a split extension over its socle; i.e. over the
product of its minimal normal subgroups. Let $F$ be a solvable Frattini-free
group of cubefree order with socle $S$. Then $S$ is a (cubefree) direct product of cyclic
groups of prime order and $F$ can be written as $F=K\ltimes S$
where $K\leq$Aut$(S)$ is determined up to conjugacy. In particular, $K$ is a subdirect product of certain
cubefree subgroups of groups of the type GL$(2,p)$ or
$C_{p-1}$. Hence in order to determine all possible subgroups $K$ one can
determine all possible projections from such a subgroup into the direct factors
of the types GL$(2,p)$ and $C_{p-1}$, and then form all subdirect
products having these projections. The construction of these subdirect
products is one of the most time-consuming parts in the Frattini extension
method for cubefree groups.
